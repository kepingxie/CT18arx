\section{\label{sec:Introduction} Introduction}

With an accumulated data sample of over 140 fb$^{-1}$ at the 13 TeV run for both ATLAS and CMS collaborations, the Large Hadron Collider (LHC) has entered an era of precision physics.
The experimental precision has been matched by improvements to the theoretical predictions, with a number of collider processes now available at the next-to-next-to-leading order
(NNLO) in the QCD coupling strength. Such precision is necessary for rigorous tests of the Standard Model (SM) and in searches for signs of physics beyond the Standard Model
(BSM), as there have been no `smoking-gun' signs of BSM physics to date. Precise predictions in QCD theory require correspondingly precise parton distribution
functions (PDFs)~\cite{Dulat:2015mca,Harland-Lang:2014zoa,Ball:2017nwa,Alekhin:2017kpj,Accardi:2016qay,Harland-Lang:2019pla,Bertone:2017bme,Manohar:2017eqh},
which in turn warrant advances in interpreting LHC experiments to extract important information about the SM, and possibly, BSM physics.

To this end, we present a new family of CTEQ-TEA parton distribution functions, designated as CT18. These PDFs are produced at both next-to-leading order (NLO) and NNLO in the QCD coupling constant, $\alpha_s$. The CT18 PDFs update those of the CT14 family presented in Ref.~\cite{Dulat:2015mca}.
In the new analysis, we include a variety of new LHC data, at the center-of-mass energies of 7 and 8 TeV, on production of single-inclusive jets, $W/Z$ bosons, and top-antitop
quark pairs, obtained by the ATLAS, CMS and LHCb collaborations. At the same time, the update retains crucial ``legacy'' data from the previous CT global QCD analyses, such as the HERA I+II
combined data on deep-inelastic scattering (DIS) and measurements in fixed-target experiments and at the Fermilab Tevatron $p\bar p$ collider. Measurements of processes in similar
kinematic regions, by both ATLAS and CMS, allow crucial cross-checks of the data. Measurements by LHCb often allow extrapolations into new kinematic regions not covered by the other
experiments. Some processes, such as $t\bar{t}$ production, allow for the measurement of multiple observables that provide similar information for the determination of PDFs. 
In addition to the PDFs themselves, we also present relevant PDF luminosities, and predictions with uncertainties for standard candle cross sections at the LHC.

The goal of the CT18 analysis is to include as wide a kinematic range for each measurement as possible, while still achieving reasonable agreement between data and theory. For the ATLAS
7 TeV jet data~\cite{Aad:2014vwa}, for example, all rapidity intervals cannot be simultaneously used without the introduction of systematic error decorrelations provided by the ATLAS
collaboration~\cite{Aaboud:2017dvo}. Even with that decorrelation, the resultant $\chi^2$ for the new jet experiments is not optimal, resulting in less effective PDF constraints. Inclusive
cross section measurements for jet production have been carried out for two different jet-radius values, $R$, by both ATLAS and CMS. For both experiments, we have chosen the data with the larger
$R$-value, for which the NNLO (fixed order) prediction should have a higher accuracy. We evaluate the jet cross section predictions using a QCD scale of inclusive jet transverse
momentum $\Q\! =\! p_T^{jet}$, consistent with past usage at NLO. The  result is largely consistent with similar evaluations using
$\Q\! =\! H_T$~\cite{Currie:2016bfm,Currie:2017ctp,Currie:2018xkj}.

Theoretical predictions for comparison to the data used in the global fit have been carried out at NNLO, either indirectly through the use of fast interpolation tables such
as \texttt{fastNLO}~\cite{Britzger:2012bs,Wobisch:2011ij} and  \texttt{ApplGrid}~\cite{Carli:2010rw}, together with NNLO/NLO $K$-factors, or directly (for top-quark related observables) through
the use of \texttt{fastNNLO} grids~\cite{Czakon:2017dip,fastnnlo:grids}. 

In an ideal world all such data sets would perfectly be compatible with each other, but differences are observed that do result in some tension between data sets and pulls in opposite directions.
One of the crucial aspects of carrying out a global PDF analysis is dealing with data sets that add some tension to the fits, while preserving the ability of the combined data set to improve on
the existing constraints on the PDFs. In some cases, a data set may be in such tension as to require its removal from the global analysis, or its inclusion only in a separate iteration of the
new PDF set. 

In this paper, we will describe how the high-precision ATLAS 7 TeV $W/Z$ rapidity distributions, which, as we find, favor an increase of the strange quark distribution at low $x$,
require such special treatment. In particular, while other PDF-analysis groups ({\it e.g.}, MMHT, see Ref.~\cite{Thorne:2019mpt}) have noted that these ATLAS $W/Z$ data can be fitted with 
$\chi^2/\mathrm{d.o.f.}$ that is comparable to the CT18 one, we find that such $\chi^2$ reflects systematic tensions with many of the other data in our global analysis.
%
%
Furthermore, the standard Hessian profiling technique used by the experimental collaborations significantly underestimates the minimal $\chi^2$ that can be reached for the ATLAS 7 TeV $W/Z$ data when they are included in the CT18 fit.
%
We therefore treat these measurements separately in an alternative fit, CT18A, introduced in Sec.~\ref{sec:alt}.
%
In another variant, CT18X, a special scale $\mu^2_{F,x} \equiv 0.8^2 \left(Q^2 + 0.3\mbox{ GeV}^2/x_B^{0.3}\right)$ is used for the calculation of low-$x$ DIS cross sections; the scale mimics the impact of low-$x$ resummation. Both modifications cause an increase in the low-$x$ quark and gluon distributions. Finally, these two variants of the CT18 fit are
amalgamated into a combined alternative fit, CT18Z, canvassed in Appendix~\ref{sec:AppendixCT18Z}, and to which we compare results throughout this article.
%
%
%

Our current global analyses are carried out in four stages. First, \texttt{PDFSense}~\cite{Wang:2018heo,Hobbs:2019gob}, a program for a rapid survey of QCD data using the Hessian
approach \cite{Pumplin:2001ct,Pumplin:2002vw}, is used to select the  data sets that are expected to have the greatest impact on the global PDF sets. This selection takes into account the sensitivity of the data to specific PDFs in a given $x$ range, 
which reflects both the correlation of these data with a given PDF, as well the size of the data set and magnitudes of its statistical
and correlated systematic errors. For example, both the collider inclusive jet data and the top-quark data have a strong correlation with the high-$x$ gluon,
but the inclusive jet data has a larger sensitivity due to a much larger number of data points. Next, \texttt{ePump}~\cite{Schmidt:2018hvu,Hou:2019gfw} is used to quickly examine the quantitative
impact of each selected data set, within the Hessian approximation. Third, the full global PDF fit is carried out using all such data sets. Recent enhancements to the CT global analysis code have
greatly improved the speed of the calculations. Lastly, the impact of key data sets on certain PDFs at specific kinematic points of interest, as well as on the value of $\alpha_s(M_Z)$,
is assessed using the Lagrange Multiplier (LM) method~\cite{Stump:2001gu}. 

In order to minimize any parametrization bias, we have tested different parametrizations for CT18: {\it e.g.}, using a more flexible parametrization for the strange quark PDF.
In some kinematic regions, there are fewer constraints from the data on certain PDFs. In these cases, LM constraints have been applied to limit those PDFs to
physically reasonable values. 

Our paper is organized as follows.
%
Section~\ref{sec:Datasets} begins with an executive summary of the key stages and results of the CT18 global analysis. It continues with an overview of the chosen experimental data
and alternative fits (CT18Z, CT18A, and CT18X) in the CT18 release.
%
%
In Sec.~\ref{sec:Theory} we detail theoretical/computational updates to the CT
fitting methodology and details for specific process-dependent calculations.
%
%
Sec.~\ref{sec:OverviewCT18} presents the main results obtained in CT18 ---
the fitted PDFs as functions of $x$ and $\Q$, determinations of QCD parameters ($\alpha_s$, $m_c$),
calculated parton luminosities, and various PDF moments and sum rules. This is the heart of the paper,
it will be of interest to a broad group of researchers who will use the PDFs for theoretical predictions at LHC experiments.
%
%
Sec.~\ref{sec:Quality} describes the ability of CT18 to provide a successful theoretical
description of the fitted data.
%
% TIM: Uncomment in v2!
%
% We also show comparisons for some data that are not included in CT18.
%
In addition to characterizing the fit of individual data sets, in Sec.~\ref{sec:StandardCandles}
we also compute the various standard candle quantities of relevance to LHC phenomenology, for
instance, Higgs boson production cross sections at $13$ and $14$ TeV, and various correlations among
electroweak boson and top-quark pair production cross sections.
%
In Sec.~\ref{sec:Conclusions}, we discuss the broader implications of this work and highlight
our main conclusions.
%
%
Several appendices present a number of important supporting details.
%
In Appendix~\ref{sec:AppendixCT18Z}, we review the CT18Z and other alternative fits,
including descriptions of various data sets admitted into these separate analyses.
%
A number of more formal details related to our likelihood functions and relations
among covariance matrices are summarized in Appendix~\ref{sec:chi2_app}.
%
Appendix~\ref{sec:AppendixParam} presents the analytical fitting form adopted in CT18
and best-fit values of the PDF parameters.
%
Appendix~\ref{sec:AppendixCodeDevelopment} presents a number of technical advances
in the CT fitting framework, including code parallelization, while Appendix~\ref{sec:ATLASjetdecorrel}
enumerates the decorrelation models utilized in fitting the newly included inclusive
jet data from the LHC.
%
%
Lastly, in Appendix~\ref{sec:Appendix4xFitter}, we present the results of a short study based on Hessian profiling
methods to assess the impact of the 7 TeV $W/Z$ production data taken by ATLAS.
