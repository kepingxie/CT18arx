\section{Discussion and conclusions
\label{sec:Conclusions}
}
In this paper, we have presented the CT18 family of parton distribution functions (PDFs), including the CT18Z, CT18A, and CT18X alternative fits. CT18 is the next generation of NNLO (as well as NLO)
PDFs of the proton from a global analysis by the CTEQ-TEA group. It represents the next update following the release of the CT14 and \CTHERAII~NNLO distributions, the latter of which was prompted
by the release of precision HERA I and II combined data after the publication of CT14. Although some of the early 7 and 8 TeV LHC Run-1 data, including measurements of inclusive production of vector bosons
\cite{Aaij:2012vn, Chatrchyan:2013mza, Chatrchyan:2012xt, Aad:2011dm} and jets \cite{Aad:2011fc, Chatrchyan:2012bja}, were included as input for the CT14 fits, CT18 represents the first CT analysis that
substantially includes the most important experimental data from the full Run-1 of the LHC, including measurements of inclusive production of vector bosons, jets, and top quark pairs at 7 and 8 TeV.
Detailed information about the specific data sets included in the CT18 global analysis can be found in Tables~\ref{tab:EXP_1} and~\ref{tab:EXP_2}, with the newly included data in the latter table.
%
%
With rapid improvements in the precision of LHC measurements, the focus of the global analysis has shifted toward providing accurate predictions in the wide range of $x$ and $Q$ covered by the LHC data, by making use of the state-of-the-art theory calculations. 
To achieve this goal requires a long-term multi-pronged effort in
theoretical, experimental, and statistical areas.

The challenge from the side of the experimental information is to select and implement relevant and consistent data sets in the global analysis. Specifically, we have included processes that have a sensitivity for the PDFs of interest, and for which NNLO predictions are available. 
For example, we include as large a rapidity interval for the ATLAS jet data as we can, using the ATLAS decorrelation model, rather than using a single rapidity interval. We noted that using a single
rapidity interval may result in selection bias. The result may be a larger value of $\chi^2 / N_\mathit{pt}$, due to remaining tensions in the ATLAS jet data, as well as reduced PDF sensitivity compared
to the CMS jet data, cf. Sec. II.B.2.
Similarly, to incorporate the $t \bar t$ differential cross section measurements into the CT18 global analysis,  
we use two $t \bar t$ single differential observables from ATLAS (using statistical correlations) and doubly-differential measurements from CMS in order to include as much information as possible. Again, there is a risk of bias if using only a single differential distribution, as some of the observables are in tension with each other, cf. Sec. II.B.3.
The CT18 global analysis shows that previous data sets, included in the CT14 global analyses, continue to have very strong pulls and tend to dilute the impact of new data. 
For example, low-energy DIS and Drell-Yan data, precision HERA data and precise measurements of the electron-charge asymmetry from D\O~ at $9.7\mbox{ fb}^{-1}$
\cite{D0:2014kma} remain important for probing combinations of quark flavors that cannot be resolved by the LHC Run 1 data alone. Furthermore, most experimental measurements contain
substantial correlated systematic uncertainties; we have taken account of these systematic errors in examining the PDF impacts of these measurements.
In addition, we have examined the PDF errors for important LHC processes and have tested the consistency of the Hessian and LM approaches.


The challenges from the theoretical side are threefold: to examine the dependence of theoretical predictions upon QCD-scale choices in comparison with experimental precision; to explore the
impact on the global analysis and
uncertainty in the chosen parametrization forms for the non-perturbative PDFs; and to be able to do fast and accurate theory calculations. In the CT18 fits, we have been using the canonical
choice of the QCD renormalization and factorization scales which typically provide the best fit to data or stabilize higher-order theoretical corrections.

For the CT18 NNLO PDFs in particular, we have consistently applied NNLO calculations to precision DIS, Drell-Yan, jet and $t \bar t$ processes, cf.~Sec.~\ref{sec:Theory}. The specific QCD-scale choices
we take for various processes are listed in Tables~\ref{Theory-Calc-II} and~\ref{Theory-Calc-VB}.
%
%
For example, a non-negligible difference was found at low-jet transverse momentum between theory predictions at NNLO using as the momentum-scale choice either the inclusive jet or the
leading-jet transverse momentum~\cite{Currie:2018xkj}. The nominal choice adopted by the CTEQ-TEA group is to use the inclusive-jet $p_T$. 
We have observed that the fitted gluon PDF is not very sensitive to this choice even in kinematic regions where the difference in predictions between these two scale choices is important.
This resilience in the global fit is due both to the presence of other data in the relevant kinematic regions, and to the QCD evolution. To compare with the high precision data at the
LHC, electroweak corrections must also be included in theory predictions. Details can be found in Sec.~\ref{sec:calcs}, cf.~Table~\ref{tab:EWcorrections}.


To examine the dependence of the fits upon the non-perturbative functional forms chosen for the PDFs at the evolution starting scale $Q_0$ (around 1.3 GeV), we have sampled a large,
$\mathcal{O}(100)$, collection of candidate fitting forms, all having a comparable number of fitting parameters. (More flexible parametrizations are used to better capture variations in
the PDFs' $x$ dependence, cf.~Appendix \ref{sec:AppendixParam}.)
The result of this study can be seen in Fig.~\ref{fig:params}.
As we increase the number of fitting parameters in the global analysis,
we typically observe a steady improvement in $\chi^2$; this improvement generally
increases so long as $\lesssim\! 30$ parameters are fitted, beyond which fits tend
to destabilize as expanded parametrizations attempt to describe statistical noise.


In order to perform the CT18 global fits at NNLO for comparison with precision data whose {\it per datum} statistical error can be as small as 0.1\%, 
we require fast theory calculations with high numerical precision. 
Hence, the usage of various fast interfaces on the calculations of structure functions and cross sections becomes mandatory and conventional.
For that, we have internally developed fast ApplGrid and FastNLO calculations at NNLO accuracy in the QCD interaction, cf.~Sec.~\ref{sec:Theory}.
In addition, we have also parallelized our global-fitting algorithms to facilitate greatly accelerated convergence times, as discussed in Appendix~\ref{sec:AppendixCodeDevelopment}.


The experimental collaborations at the LHC have succeeded in taking copious high-precision data. To examine
the agreement with these precision data calls for advances in statistical methodology. 
Which of the eligible LHC experiments provide promising constraints on the CTEQ-TEA PDFs?
Do the LHC experiments agree among themselves and with other experiments?
A consistent answer emerges from a powerful combination of four methods:
1) \texttt{PDFSense} and $L_2$ sensitivity,
2) the \texttt{ePump} program, 
3) Effective Gaussian variables,
and 4) LM scans.
While the last two methods had been introduced in the previous CTEQ-TEA global analysis, such as CTEQ6, CT10 and CT14, the first two techniques were invented in the process of CT18 global analysis. 

The \texttt{PDFSense} program~\cite{Wang:2018heo} provides an easy way to visualize the potential impact of data on PDFs in the $x$ and $Q$ plane. In addition, a simple $L_2$ sensitivity variable \cite{Hobbs:2019gob} is instructive for exploring agreement between different experiments similarly to the LM scans, but using a much faster Hessian formalism across the full range of $x$ or $Q$. See examples in Secs.~\ref{sec:L2} and \ref{sec:AppendixCT18Z}. 

The complementary \texttt{ePump} program~\cite{Schmidt:2018hvu} contains a fast and efficient method to estimate the effect of new data on a set of best-fit and Hessian error PDFs. Extensive validations against the previous CT14 global fits have also been performed~\cite{Hou:2019gfw}.
The application of the above four techniques in the CT18 analysis is illustrated throughout this paper. 

With this methodology, we performed the global analysis and produced the CT18 PDFs and the alternatives, CT18A, CT18X and CT18Z.
(See Sec.~\ref{sec:alt} and Table~\ref{tab:AXZ}.) CT18 represents the nominal CTEQ-TEA PDF set; CT18A is the product of adding the ATLAS 7 TeV $W/Z$ data~\cite{Aaboud:2016btc})
into the CT18 fit; CT18X is a variation of CT18 with an $x$-dependent QCD scale for the low-$x$ DIS data (along with a slightly larger charm quark mass value of 1.4 GeV);
and CT18Z contains all the above variations and generally differs most significantly from CT18. 


In the CT18 analysis, important impacts are found on PDFs from ATLAS and CMS inclusive jet production measurements, LHCb $W$ and $Z$ vector boson productions and ATLAS $\sqrt{s}=8$ TeV $Z$ boson transverse momentum data.
We find contradictory preferences for the strange quark PDF between semi-inclusive (SI) DIS (e.g., NuTeV and CCFR dimuon production) experiments, on one hand, and some LHC experiments, especially ATLAS 7 TeV $W/Z$ production measurements and to some extent LHCb $W/Z$ measurements, on the other hand. Benchmarking of LHC measurements and theoretical predictions, as well as new (SI)DIS experiments can be highly effective for resolving these tensions. Going forward, to
facilitate the discovery program of the HL-LHC, a sustained effort to navigate experimental tensions in collider data will be required to
achieve the ultimate precision of these planned experiments.  We envision an interplay among theoretical and data-analytical methods (including those
used in this study to explore data compatibility), and additional high-precision experiments such as high-luminosity DIS colliders like the
Electron-Ion Collider (EIC) \cite{Accardi:2012qut}, to be indispensable for making such progress.


 

The inclusion of new data and theoretical advances have resulted in the following changes in CT18, as compared to CT14: 
1) a smaller $g(x,Q)$ for $x \sim 0.3$ (mainly due to ATLAS and CMS 7 TeV jet data and ATLAS 8 TeV $Z$ $p_T$ data, with some tension found between CMS 7 and 8 TeV jet data); 
2) some changes in $u$, $d$, $\bar u$ and $\bar d$ at small $x$, such as a larger $d$ and $d/u$ and a smaller $\bar d / \bar u$ for $x \sim 0.2$ (mainly due to LHCb $W$ and $Z$ rapidity data and CMS 8 TeV $W$ lepton charge asymmetry data); and 
3) a larger $s$ and $(s+\bar s)/(\bar u + \bar d)$ at small $x$ (mainly due to LHCb $W$ and $Z$ rapidity data, and further enhanced by the ATLAS 7 TeV $W/Z$ data in the CT18A and CT18Z fits).
While the sensitivity of an individual $t \bar t$ data point can be similar to that of an individual jet data point at the LHC, the total sensitivity of the $t \bar t$ data is small due to the small number of  $t \bar t$ data points. Hence, we did not find noticeable impact from the double differential distributions of the $t \bar t$ data included in the CT18 analysis. A similar finding was also reported in Ref.~\cite{Hou:2019jxd}, in a CT14 analysis. 

Despite these changes in central predictions, the CT18 NNLO PDFs remain consistent with CT14 NNLO within the respective error bands.
More details about the comparison of CT18 and CT14 PDFs, as well as the quality of the fits to data can be found in Sections IV and V. 

Some implications of CT18 predictions for phenomenological observables
were reviewed in Sections V and VI.
Compared to calculations with CT14 NNLO, both the $gg \rightarrow H$ and $t\bar t$  total
cross sections have decreased slightly in CT18. The $W$ and $Z$
cross sections, while still consistent with CT14, have slightly increased as a result of enhanced strangeness. Common ratios of
strange and non-strange PDFs for CT14 NNLO, shown in Sec. IV, are consistent with
the independent
ATLAS and CMS determinations within the PDF uncertainties.

We have also presented the implications of the CT18 global fits for the value of $\alpha_s$, as seen in Sec.~\ref{sec:AlphasDependence}.   
The full CT18 data set prefers, at NNLO, a value of $\alpha_s(M_Z)\! =\! 0.1164\! \pm\! 0.0026$, at 68\% C.L. The corresponding value for CT18Z is basically the same, $0.1169\pm0.0027$.
This is to be compared to the CT14 determination, which included very little LHC data, of $\alpha_s(M_Z)\! =\! 0.115^{+0.006}_{-0.004}$ at 90\% C.L.

The LM scans over the charm quark (pole) mass, $m_c$, as shown in Figs.~\ref{fig:lm_mc} and~\ref{fig:lm_mcZ}, support the usage of 1.3 GeV and 1.4 GeV in the CT18 and CT18Z fits,
respectively. Notably, the combined HERA charm data prefer a somewhat smaller $m_c$ value, while the ATLAS 7 TeV $W/Z$ data in the CT18Z fit prefer a larger $m_c$ value. Comments about the impact of fitted charm contributions on predictions for LHC $W/Z$ cross sections are made at the end of Sec.~\ref{sec:ellipse}.
Comparisons to the parton luminosities and predictions based on the PDFs from other groups can be found in Secs.~\ref{sec:PDFLuminosities}, \ref{sec:moments}, and \ref{sec:ATL7ZWchi2}.

To allow direct comparison to results obtained by the lattice QCD community, we have also presented the CT18 predictions for various PDF moments and sum rules in Sec.~\ref{sec:moments}.
In general, we find good agreement between CT18 and results from other phenomenological fitting efforts for most lattice observables.  At present, systematic
effects are such that many lattice calculations significantly overshoot the predictions of contemporary phenomenology, with the exception of the gluonic moment $\langle x \rangle_g$, which
is underpredicted by the lattice relative to PDF fits.  We expect complementary advances in lattice simulations and PDF phenomenology to improve this situation
in coming years and pave the way for a synergistic PDF-Lattice effort \cite{Hobbs:2019gob,Lin:2017snn} to determine the nucleon's longitudinal structure.

The final CT18 PDFs are presented in the form of 1 central and 58
Hessian eigenvector sets at NNLO and NLO. The 90\% C.L. PDF
uncertainties for physical observables can be estimated from these
sets using the symmetric \cite{Pumplin:2002vw} or asymmetric
\cite{Lai:2010vv,Nadolsky:2001yg} master formulas by adding
contributions from eigenvector pairs in quadrature. These PDFs are
determined for the central QCD coupling of $\alpha_s(M_Z)=0.118$,
consistent with the world-average $\alpha_s$ value. For estimation of
the combined PDF+$\alpha_s$ uncertainty, we provide two additional
best-fit sets for $\alpha_s(M_Z)=0.116$ and 0.120. The
90\% C.L. variation due to $\alpha_s(M_Z)$ can be estimated as a one-half of the
difference in predictions from the two $\alpha_s$ sets. The
PDF+$\alpha_s$ uncertainty, at 90\% C.L., and including correlations,
can also be determined by adding the PDF uncertainty and $\alpha_s$
uncertainty in quadrature~\cite{Lai:2010nw}.
%
%At leading order, we provide two PDF sets,
%obtained assuming 1-loop evolution of $\alpha_s$ and
%$\alpha_s(M_Z)=0.130$; and 2-loop evolution of $\alpha_s$
%and $\alpha_s(M_Z)=0.118$.
%
Aside from these general-purpose PDF sets, we provide a series of (N)NLO
sets for $\alpha_s(M_Z)=0.111-0.123$
and additional sets using heavy-quark
schemes other than our standard 5-flavor method, with up to 3, 4, and 6 active flavors. 

Parametrizations for the CT18 PDF sets are distributed in a standalone
form via the CTEQ-TEA website \cite{CT18website}, or as a part of
the LHAPDF6 library \cite{LHAPDF6}. For backward compatibility with
version 5.9.X of LHAPDF, our website also provides CT18 grids in the
LHAPDF5 format, as well as an update for the CTEQ-TEA module
of the LHAPDF5 library, which must be included during compilation
to support calls of all eigenvector sets included with CT18 \cite{LHAPDF5}.



\begin{acknowledgments}
	%
We are indebted to our friend and colleague Jon Pumplin for decades of fruitful collaboration and for his last major contribution to the CTEQ-TEA global analysis made in this article.
We thank Stefano Camarda, Amanda Cooper-Sarkar, Alexander Glazov, Lucian Harland-Lang, Jan Kretzschmar, Bogdan Malescu, Wally Melnitchouk, Dave Soper, and CTEQ colleagues for insightful discussions. This work is partially supported by the U.S.~Department of Energy under Grants No.~DE-SC0010129 (at SMU) and No.~DE-FG02-95ER40896 (at U. Pittsburgh); by the U.S.~National Science Foundation
under Grant No.~PHY-1719914 (at MSU) and No.~PHY-1820760 (at U.~Pittsburgh), and in part by the PITT PACC. T.~J.~Hobbs acknowledges support from a JLab EIC Center Fellowship.
The work of M.Guzzi is supported by the National Science Foundation under Grant No.~PHY-1820818. The work of J.G.~was supported by the National Natural Science Foundation (NNSF)
of China under Grants No.~11875189 and No.~11835005, and the work of S.D.~and I.S.~under NNSF Grants No.~11965020 and No.~11847160. C.-P.~Yuan is also grateful for the support from
the Wu-Ki Tung endowed chair in particle physics.

\end{acknowledgments}